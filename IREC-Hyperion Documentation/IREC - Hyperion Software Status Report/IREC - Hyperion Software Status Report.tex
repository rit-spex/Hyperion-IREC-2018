\title{IREC - Hyperion Software Status Report}
\author{
	Dylan Wagner (drw6528@rit.edu) \\
	RIT Space Exploration
}

\date{\today}

\documentclass[11pt]{article}
\usepackage{hyperref} 
\hypersetup{
    colorlinks=true,
    linkcolor=blue,
    filecolor=magenta,      
    urlcolor=cyan,
}
\urlstyle{same}

\begin{document}
\maketitle

\section{Work Completed}
This section will outline work completed on the system software of the Hyperion payload. 
\begin{itemize}
  \item Inital testing of hardware modules:
  \begin{itemize}
  	\item LSM9DS1 (Accelerometer , Gyroscope, Magnetometer) Sensor
	\item BME280 (Pressure, Humidity, Temperature) Sensor
	\item CCS811 (TVOC, CO2) Sensor
	\item LIS331 (Accelerometer) Sensor
	\item RFM\_9X LoRa 
  \end{itemize}
  \item Utility functions for hardware modules
  \begin{itemize}
   	\item LSM9DS1 (Accelerometer , Gyroscope, Magnetometer) Sensor
	\item BME280 (Pressure, Humidity, Temperature) Sensor
	\item CCS811 (TVOC, CO2) Sensor
	\item LIS331 (Accelerometer) Sensor
	\item RFM\_9X LoRa 
  \end{itemize}
  \item Defining a communication protocol for the Hyperion Payload
  \item Implementing the payload communication protocol
  \item Creation of a robust execution scheduling queue (\href{https://github.com/RIT-Space-Exploration/Dynamic-Scheduling-Queue}{Dynamic Scheduling Queue})
  \item Forked RadioHead library to work with Teensy3.6
  \item Tested:
  \begin{itemize}
  	\item Communication between RFM\_9X modules 
	\item Communication between RFM\_9X modules using the Hyperion payload Protocol
	\item The Dynamic Scheduling Queue
	\item Communication routines within the the system
  \end{itemize}	
\end{itemize}

\section{In Progress / ToDo}
This section will outline work that is in progress or will be in the near future. 
\begin{itemize}
	\item Further defining routines within the system
	\item Implementation of routines regarding:
	\begin{itemize}
		\item Gathering data from sensors, logging it
		\item Transmission of data groups with the Hyperion Protocol
		\item Writing the data buffer to the storage card
		\item Mission Events (Parachute deployment, Impact damper deployment)
		\item Maintaining impact damper pressure
	\end{itemize}
	\item Testing implemented routines
	\item Initial testing of the Pressure Transducer 
	\item Utility functions for the Pressure Transducer 
	\item Further optimizing the Hyperion Protocol for low-bandwidth environments
	\item Testing deployment sensing mechanism 
	\item Testing software with mission event tests (Parachute deployment, Impact damper deployment)
	\item Readjustment of static priority values within the Dynamic Scheduling Queue.
	\item Testing of overall system
	\item Data feeder software for ground station:
	\begin{itemize}
		\item Software to collect transmissions from payload on a micro-controller 
		\item Software to collect data over serial from connected micro-controller and feed data to HabNet
	\end{itemize}
\end{itemize}

\section{Links}
\begin{itemize}
	\item The Rochester Institute of Technology Space Exploration Website
	\\ \url{http://spex.rit.edu}
	\item The Rochester Institute of Technology Space Exploration GitHub Page \url{https://github.com/RIT-Space-Exploration}
	\item IREC - Hyperion Main Repository (GitHub)
	\\ \url{https://github.com/RIT-Space-Exploration/IREC-Hyperion}
	\item Dynamic Scheduling Queue (GitHub)
	\\ \url{https://github.com/RIT-Space-Exploration/Dynamic-Scheduling-Queue}
	\item Forked RadioHead Library (GitHub)
	\\ \url{https://github.com/RIT-Space-Exploration/RadioHead}
\end{itemize}

\end{document}
